% Created 2024-06-07 Fri 11:30
% Intended LaTeX compiler: pdflatex
\documentclass[11pt]{article}
\usepackage[utf8]{inputenc}
\usepackage[T1]{fontenc}
\usepackage{graphicx}
\usepackage{longtable}
\usepackage{wrapfig}
\usepackage{rotating}
\usepackage[normalem]{ulem}
\usepackage{amsmath}
\usepackage{amssymb}
\usepackage{capt-of}
\usepackage{hyperref}
\usepackage[backend=bibtex]{biblatex}
\addbibresource{/home/laurent/Documents/code/hera-mi-test/report/myrefs.bib}
\usepackage{fancyvrb}
\DefineVerbatimEnvironment{verbatim}{Verbatim}{framerule=0.5mm,frame=lines,numbers=left}
\bibliography{myrefs.bib}
\author{Laurent Lejeune}
\date{\today}
\title{A Deep-Learning Approach to Breast Cancer Screening from Mammography Images}
\hypersetup{
 pdfauthor={Laurent Lejeune},
 pdftitle={A Deep-Learning Approach to Breast Cancer Screening from Mammography Images},
 pdfkeywords={},
 pdfsubject={},
 pdfcreator={Emacs 29.3 (Org mode 9.6.24)}, 
 pdflang={English}}
\begin{document}

\maketitle
\tableofcontents


\section{Mammography Dataset}
\label{sec:orgc746c8a}

We use the publicly available Chinese Mammography Database (CMMD) \autocite{cai23},
which originally contains \(\approx 1871\) patients screened for breast cancer.
and apply relevant filtering criteria to remove:
\begin{itemize}
\item Patients with history of previous breast biopsy within 1 week, or any therapy for breast lesions
prior to mammography
\item Patients with breasts prosthesis
\item Images with substantial motion artifact
\end{itemize}

Each patient is then diagnosed by an expert and assigned the following target variables:
\begin{itemize}
\item \(y_{p} \in \{\text{benign}, \text{malignant}\}\) indicates the type of tumor.
\item \(y_{a} \in \{\text{calcification}, \text{mass}, \text{both}\}\) indicates the type of abnormality, where \texttt{both} means that both \texttt{calcification} and \texttt{mass} are present.
\item \(y_s \in \{\text{luminal-A}, \text{luminal-B},\text{HER2-positive},\text{triple-negative},\text{missing}\}\) a
subtype information (possibly missing).
\end{itemize}



\section{Dataset exploration}
\label{sec:orgfbd253f}

\printbibliography
\end{document}
